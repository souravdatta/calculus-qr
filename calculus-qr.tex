\documentclass[a4paper]{book}
\title{\LARGE{Calculus Quick Reference}}
\author{Sourav Datta, et al}
\date{}
\begin{document}
\maketitle

\part{Before Calculus}

\chapter{Some Useful Algebraic Formulas}

\section{Arithmetic and Geometric Progressions}
Let $S$ be the sum of a series and $l$ be the last element of the series. 
For the Arithmetic Progression for a series of $n$ terms starting with $a$ and having a difference between the terms $n+1$ and $n$ denoted as $t_{n+1} = t_{n} + d$
\begin{equation}
S = \frac{n}{2}[2a + (n - 1)d]
\end{equation}
\begin{equation}
l = a + (n - 1)d
\end{equation}
Special case when $a = 1$ and $d = 1$,
\begin{equation}
S = \frac{n}{2}(n + 1)
\end{equation}
\begin{equation}
l = n
\end{equation}
For the Geometric Progression for a series of $n$ terms starting with $a$ and having a difference between the terms $n+1$ and $n$ denoted as $t_{n+1} = t_{n}r$
\begin{equation}
S = \frac{a(r^{n} - 1)}{r - 1}
\end{equation}
\begin{equation}
S = \frac{a(1 - r^{n})}{1 - r}
\end{equation}
\begin{equation}
l = ar^{n - 1}
\end{equation}

\section{Roots of Quadratic Equations}
The roots of a quadratic equation of the form $ax^{2} + bx + c = 0$ is given by,
\begin{equation}
x = \frac{-b \pm \sqrt{b^{2} - 4ac}}{2a}
\end{equation}
Properties of the roots
\begin{description}
\item If $b^{2} - 4ac$ is $positive$, roots are real and unequal.
\item If $b^{2} - 4ac$ is $0$, roots are real and equal ($= \frac{b}{2a}$).
\item If $b^{2} - 4ac$ is $negative$, roots are imaginary and unequal.
\item If $b^{2} - 4ac$ is $perfect square$, roots are rational and unequal.
\end{description}


\section{Permutations and Combinations}
\textbf{Permutation} is the process of $arranging$ some or all elements from a set. \textbf{Combination} is the $grouping$ or $selection$ of some or all elements from a set. 
\begin{enumerate}
\item If an operation can be performed $m$ ways and another operation can be performed $n$ ways, then the number of ways the two operation can be performed (in any order) is $mn$.
\item Find the $Permutations$ of n $unique$ elements taken $r$ at a time
\begin{equation}
^{n}P_{r} = n(n - 1)(n - 2)...(n - r + 1)
\end{equation}
\item Find the $Combinations$ of n $unique$ elements taken $r$ at a time
\begin{equation}
^{n}C_{r} = \frac{n!}{r!(n - r)!}
\end{equation}
\item The number of combinations of $n$ elements taken $r$ at a time is equal to number of combinations of $n$ elements taken $n - r$ at a time. This is called complementary combination
\begin{equation}
^{n}C_{r} = ^{n}C_{n - r}
\end{equation}
\item The number of ways $m + n$ $unique$ objects can be divided in $m$ and $n$ groups
\begin{equation}
N_{m,n} = \frac{(m + n)!}{m!n!}
\end{equation}
If $m = n$, then
\begin{equation}
N_{m,n} = \frac{(m + n)!}{m!m!2!}
\end{equation}
\item The number of ways $m + n + p$ $unique$ objects can be divided in $m$, $n$ and $p$ groups
\begin{equation}
N_{m,n,p} = \frac{(m + n + p)!}{m!n!p!}
\end{equation}
If $m = n = p$, then
\begin{equation}
N_{m,n,p} = \frac{(m + n + p)!}{m!m!p!3!}
\end{equation}
\item The number of ways $n$ objects can be $arranged$ among themselves, where $p$ objects are of one kind, $q$ objects are another kind, $r$ objects are of third kind and rest are different 
\begin{equation}
x = \frac{n!}{p!q!r!}
\end{equation}
\end{enumerate}

\section{Binomial Theorem}
Binomial theorem for a constant $a$ and any index $n$ is defined as,
\begin{equation}
(x + a)^{n} = x^{n} + ^{n}C_{1}ax^{n - 1} + ^{n}C_{2}a^{2}x^{n - 2} + ... + ^{n}C_{n}a^{n}
\end{equation}

\section{Inequalities}
\begin{description}
\item If $a < b$ then,
\begin{equation}
a + c < b + c
\end{equation}
\begin{equation}
a - c < b - c
\end{equation}
when $c > 0$
\begin{equation}
ac < bc
\end{equation}
\begin{equation}
\frac{a}{c} < \frac{b} {c}
\end{equation}
when $c < 0$,
\begin{equation}
ac > bc
\end{equation}
\begin{equation}
\frac{a}{c} > \frac{b} {c}
\end{equation}

\item If $a > b$ then,
\begin{equation}
a + c > b + c
\end{equation}
\begin{equation}
a - c > b - c
\end{equation}
when $c > 0$
\begin{equation}
ac > bc
\end{equation}
\begin{equation}
\frac{a}{c} > \frac{b} {c}
\end{equation}
when $c < 0$,
\begin{equation}
ac < bc
\end{equation}
\begin{equation}
\frac{a}{c} < \frac{b} {c}
\end{equation}
\item If $a > b$
\begin{equation}
-a < -b
\end{equation}
\item If $a < b$
\begin{equation}
-a > -b
\end{equation}
\end{description}
The same inequalities hold when the signs contain an equal component as well.


\chapter{Trigonometry}
This chapter briefly reviews the basics of Trigonometry. Calculus applied on Trigonometry will be discussed in later chapters.


\end{document}
